\documentclass[a4paper, 11pt]{scrartcl}

\usepackage[english]{babel}
\usepackage{csquotes}
\usepackage[right=4.5cm]{geometry}
\usepackage{setspace}
\usepackage{hyperref}

\onehalfspacing

%%%%   ARGUMENTS
\newenvironment{argument}{
	\begin{flushleft} \vspace{5mm}
		\begin{tabular}{p{1em} p{0.08\linewidth} p{0.75\linewidth}}
		}
		{
		\end{tabular}
	\end{flushleft}
}

\newcommand{\premise}[2]{& #1: & #2 \\[0.4em]}
\newcommand{\premiseEq}[3]{& #1: \newline {\footnotesize =#2} & #3 \\[0.4em]}
\newcommand{\intermediate}[3]{& #1: & \emph{Therefore}: #2 (#3) \\[0.4em]}
\newcommand{\conclusion}[3]{\cline{2-3} \\[-0.8em] & #1: & \emph{Therefore}: #2 (#3)\\[0.4em]}


\title{Your Title}
\author{Your Name}

\makeindex

\begin{document}

\maketitle
\begin{center}
	\small (XXXX words)
\end{center}

\section{This Is a Simple Template}

This is a template for your essay. It is an ordinary \texttt{scrartcl} document with a little environment for arguments and a formatting that makes corrections easier for us. You can, but do not have to use it. Feel free to adopt it to your needs.

\section{Argument Example}

\begin{argument}
	\premise{P1}{This is the first premise.}
	\premise{P2}{And this is the second premise.}
	\premiseEq{P3}{C3}{The third premise is equal to the conclusion of another argument.}
	\intermediate{C1}{This is our first intermediate conclusion, which follows from P1, P2, and P3.}{P1 -- P3}
	\premise{P4}{The fourth premise.}
	\intermediate{C2}{And the second intermediate conclusion, following from C1 and P4.}{C1, P4}
	\premise{P5}{The final premise.}
	\conclusion{C}{Finally, our conclusion.}{C2, P5}
\end{argument}

\end{document}